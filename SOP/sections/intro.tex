As a volunteer for caring autism teenagers, my close encounter with them led to my interest in \textit{Social Computing}. 
Autism reduces social awareness and language abilities, 
making it hard to emotionally connect with the sufferer through normal communication channels.
Online interaction offers an alternative approach to bridge their social gap.
During the volunteer days, I was impressed by the active social characteristics of several autism teenagers online,
which is just opposite to their behaviors in reality.
One of the most impressive experiences is an online conversation between me and a high-functional autism boy,
who hardly responsed in face-to-face interaction.
I was surprised that he actively opened an online chat with me to invite me to take a picnic with him and his family.
Although his words were ill organized, a positive personality was conveyed via typed messages.
The mediation of computers turns our communication into a channel which is more perceptable.\\

\noindent
Witnessing the effect of the Internet on the boy, I asked myself:
``how to help humankind apply to and live better with cyber communication channel?''
Similar changes are not unique to special groups but are encountered routinely in daily life,
such as one's hyperpersonality online.
Technology has extended the social space from reality to virtual world.
People now can open social interaction in any time and any places, overcoming the temporal and spatial obstacles.
With the increasing growing rate of the Internet and technologies, I believe that in the following decades, 
\textit{Social Computing} will remain an important research domain under \textit{Human-Computer Interaction} (\acr{HCI}),
especially \textit{Computer-Mediated Communication}, \textit{Cross-culture Online Interaction}, and \textit{Crowdsourcing}.
I want to develop emerging technologies mediating social interactions and stimulates the potential power of crowds. 
Ultimately, I aim to continue my research in academy and nourish the future stars after pursuing Ph.D.\\




