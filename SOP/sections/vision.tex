%\noindent As a computer science student, I asked myself: 
%``can technology mediate communication by enhancing expression and perception of social behaviors?''
%Thanks to my high social ability, I can communicate with the autistic kid in the end. 
%However, not everyone has such high social ability. 
%Similar problematic communication do not only exist in special groups, but also in our daily lives. 
%It deteriorates our relationship, health, and working performance. 
%Nowadays, computers exist around us. 
%They can detect the tiny change from human and send messages in a cross-modal form, 
%helping us better perceive and express the social interaction status. 
%With enhanced perception and expression, people can adapt their behaviors accordingly. 
%A benign cycle of communication between human, computers, and human can therefore be formed. 
%To conduct the research, 
%I need a highly interdisciplinary environment to provide cross-domain knowledge of computer science, design, and psychology. 
%SOME PROGRAM at SOME SCHOOL emphasizes the insights from diverse domain. 
%Your cross-domain collaboration and resources can best nourish my research on CSCW and Social Computing.
\noindent
My ultimate ambition is to become a professor to nourish social computing research in Taiwan.
Ideas like \textit{Spirit Bomb} are hardly found in Western comics, which might present Eastern's collectivism.
Similarly, I have long been wondered if cultural difference has anything to do with the state of the art of social computing. 
As an Asian, it is my inherent responsibility to bridge the gap.
Thus, an environment with high interdisciplinary and multi-culture is necessary
for me to earn cross-domain knowledge of computer science, design, and psychology.
SOME PROGRAM at SOME SCHOOL emphasizes the insights from diverse domain. 
Your cross-domain collaboration and resources can best nourish my research on \textbf{Social Computing}.\\



