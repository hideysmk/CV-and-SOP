My close encounter with with children with autism led to my interest in \textit{Social Computing}. 
Autism reduces social awareness and language abilities, 
making it hard to emotionally connect with the sufferer through normal communication channels.
I have been a volunteer to teach autistic kids self-care skills for several years.
Initially I had no idea how to communicate with them, 
but my social prowess soon helped me realize that their actions had significance. 
For example, there was a boy beat his head whenever he was playing games. 
To most people this indicates frustration, while in his case it indicated satisfaction.
However, communication gaps are not unique to special groups but are encountered routinely in daily life. 
Poor communication impacts on relationships, work performance and even health. \\

\noindent
As a computer science student, I asked myself: 
``can technology mediate communication by enhancing perception and expression of social behaviors?'' 
In this era of flat world, people interact and work together via latest technology.
Computers mediate communication in a cross-modal form, helping us better perceive and socially interact. 
With bidirectionally enhanced interaction people can adapt their behaviors according to the context. 
A benign cycle of communication, based on interaction between humans and computers, can be established. 
I believe that Social Computing, 
especially \textit{Computer-Mediated Communication} (\acr{CMC}), \textit{Cross-culture Online Interaction}, and \textit{Crowdsourcing},
will remain important research topics in Human-Computer Interaction (\acr{HCI}) in the near future.
I want to develop technologies mediating human social interaction by enhancing perception and expression of social behaviors.
Ultimately, I aim to conduct innovative research in academy after pursuing Ph.D.\\

