\noindent
As soon as I completed my compulsory military service, 
I continue my research in Intel-National Taiwan University (\acr{NTU}) research center, 
working on research of communication involving multiple users and intelligent machines. 
Besides, I am working on friendsourcing with Prof. Hao-Chuan Wang of National Tsing-Hua University, 
who establishes the first lab for social computing in Taiwan. 
I beleive that social relationships can be an excellent source of motivation to mobilize people for quality crowd-based works. 
In an initial study, we found that requesters are willing to pay more than what workers expect in friendsourcing, 
which is counter-intuitive as crowdsouricng researchers and practitioners tend to consider friends' help free. 
This primitive finding has been now submitted as a poster to \emph{CSCW '15}. 
We are now working on a hybrid crowd-powered document edting system, 
which combines regular crowdsourcing and friendsourcing in order to direct microtasks of different properties and requirements. 
For example, grammar checking requires only language skills while editing of personal content may work better when social relations exist. 
A system that carefully blends and leverages regular crowdsourcing and friendsourcing is likely to generate value that each of them cannot obtain.\\

\noindent
From the research experience above, 
I realized that computing machines alone are not enough to realize computer-mediated communication of satisfying experience of the users. 
These computer-based communication channels need elegant design to better integrate them with the communication processes. 
Thus I joined the Reminiscence-Aiding Interface project.
I and my colleagues explored the effects of soundscape in aiding reminiscence, i.e. recalling and telling past stories. 
The series of the soundscape project were accepted as posters in \emph{ACM DIS '12} and \emph{ACM CHI '13}, and as a full paper in \emph{IASDR '13}.\\


\noindent
As the first step, 
I have been working on crafting intelligent machines to interpret human behaviors. 
I started working on social signal processing of face-to-face conversational engagement when pursuing my master's degree, 
advised by Prof. Jane Hsu in Intelligent Agents Lab in NTU. 
The traditional approach of social signal analysis used sophisticated mathematical models to model and interpret social behaviors. 
I proposed a more flexible framework to involve human intelligence for assessing face-to-face conversational engagement. 
With Coupled Hidden Markov Model, the approach reached 82\% accuracy of engamement identification with respect to expert-labeled data. 
The work was accepted by AAAI-12 Activity Context Representation Workshop, and developed into my master’s thesis eventually.\\

\noindent
I also had the opportunity to visit Carnegie Mellon University (\acr{CMU}) Silicon Valley Campus for collaborative research between NTU and CMU. 
I joined a project led by Prof. Ted Selker on Auditory Presented Social Information. 
We aimed to understand how audio channel presents three pieces of social information: 
speaker identity, presence, and entry/exit, in teleconference system. 
I carefully designed the icons according to syntax (sound and placement) and semantic (relationship to the conversational channel). 
The work successfully showed that information from audio channel can improve users' sensory experience in the interaction. 
The paper was accepted by \emph{INTERACT '13}. \\





%\noindent
%Modern computing devices are no more ``just one of communication channels''.
%Intelligent and ubiquitous computation makes computers possible to process information for communication participants,
%and actively participate in the interaction.
%I would describe these machines as ``implicit participants'',
%and the interaction is the interplay between mulitple human and multiple devices.
%However, social mediation that current computing devices can actively provide is still limited.
%My prior experiences has provided me several pieces of the puzzle,
%but to collect more pieces and aggregate them into a whole picture, 
%pursuing a PhD is necessary to me.\\
%
%\noindent
%As the first step, I started working on crafting intelligent machines to interpret human behaviors.
%I started working on social signal processing of \textit{Face-to-Face Conversational Engagement} when pursuing master's degree,
%advised by Prof. Jane Hsu in Intelligent Agents Lab, National Taiwan University (\acr{NTU}).
%The traditional approach of social signal analysis is using sophisticated mathematical model to interpret social behaviors from digital signals.
%I proposed a framework which are flexible to involve human intelligence for assessing face-to-face conversational engagement.
%With Coupled Hidden Markov Model, 
%the machine-assessed engagement level received an 82\% accuracy on expert-labeled data. 
%The work was accepted by \textit{AAAI-12 Activity Context Representation Workshop}, and developed into my master's thesis eventually.\\
%% Oct. 28: remember to modify this part as to add "crowds' power"
%
%\noindent
%As the most active lab member,
%Prof. Jane Hsu recommended me to visit Carnegie Mellon University Silicon Valley for cooperative research.
%I joined a project led by Prof. Ted Selker on \textit{Auditory Presented Social Information}.
%We aimed to understand how audio channel presents three pieces of social information: 
%speaker identity, presence, and entry/exit, in teleconference system.
%I carefully designed the icons according to syntax (sound and placement) and semantic (relationship to the conversational channel).
%Successfully showing that information from audio channel can improve users' sensory experience in the interaction, 
%the paper was accepted by \textit{INTERACT '13}. 
%The knowledge of presenting social information in audio channel thus became the next piece that I have collected. \\
%
%
%\noindent
%From the research expericnce above, 
%I realized that high-functional computing machines are not enough to form computer-mediated communications with good user experiences.
%These ``implicit particpants'' need a design for elegant interaction to integrate them into the process of communications.
%Thus I joined the \textit{Reminiscence-Aiding Interface} project, which is a cooperative project led by Prof. Rung-Huei Liang.
%I and my colleagues explored the potential effects of soundscape in aiding reminiscence, i.e. recalling and telling past stories.
%The series of the soundscape project were accepted as posters in \textit{DIS '12} and \textit{CHI '13}, 
%and as a full paper in \textit{IASDR '13}.\\
%
%
%\noindent
%My enthusiasm for research is not reduced even after the one-year blank for military service.
%As soon as I completed my conscription, I continue my research in Intel-NTU research center,
%working on research of communication involving multiple users and intelligent machines.
%Besides, I am working on friendsourcing with Prof. Hao-Chuan Wang, who establishes the first lab for social computing in Taiwan.
%I beleive that social relationships can be an excellent incentive to mobilize people for quality crowd-based works.
%We found that requesters are willing to pay more than workers expect in friendsourcing,
%which is a counter-intuitive phenomenon since people usually consider friends' help to be free.
%This primitive finding is now submitted as a poster to \textit{CSCW '15}.
%We are now working on designing \textit{Social-Powered SOP} (\acr{SOPSOP}),
%a social-driven proofreading and editting system for statement of purpose, aiming to further utilizing friendsourcing. \\
%
%
%
%
