\noindent
My research explores novel technologies and techniques aiming to explore and improve social interaction. % modify "'aid' social interaction"
My expertise in CS was cultivated in National Taiwan University (\acr{NTU}),
the top one university and its Dept. of CS in Taiwan. 
Currently I am an user experience (\acr{UX}) research assistant of \textit{Intel-NTU Research Center}, 
focusing on UX research on internet of things.
Meanwhile, 
I am also a collaborative researcher in \textit{Communication and Social Computing Lab} of Dept. CS of National Tsing-Hua University,
the first lab researching on Social Computing in Taiwan.\\

\noindent
\textbf{Social Signal Processing}:
I worked on Social Signal Processing (\acr{SSP}) when pursuing master's degree, 
proposing a framework to improve the process of Conversational Engagement Analysis. 
With Coupled Hidden Markov Model (\acr{CHMM}), 
the machine-assessed engagement level received an 82\% accuracy on expert-labeled data. 
The early version of the work was accepted by \textit{AAAI-12 Activity Context Representation Workshop}. 
The chance of presenting and discussing the research idea to global researchers inspired me to conduct following research, 
which led to the final version of my master’s thesis.\\

\noindent
\textbf{Intelligent Teleconference System}:
To augment CMC, I explored audio channel under the supervision of Professor Ted Selker. 
In 2012 summer, Prof. Selker recruited five students from NTU for short-term research.
I was the only student selected to join the CAMEO project, working on designing intelligent teleconference system. 
We used auditory icons to present three pieces of social information: speaker identity, presence, and entry/exit. 
I carefully designed the icons according to syntax (sound and placement) and semantic (relationship to the conversational channel). 
Successfully showing that utilizing audio channel can improve users' sensory experience in the interaction, 
the paper was accepted by \textit{INTERACT '13}. 
In addition to the achievement of research, 
the experience also helped me develop the ability of overcoming the challenge of facing new research topic with new colleagues in a new environment. 
After the CMUSV days, I am more confident that pursuing PhD will be not only necessary but also a correct choice to my life.\\

\noindent
\textbf{Friend Sourcing}:
Currently, I am working on Community Sourcing.
In community sourcing, 
requesters and workers share common traits and have closer relationship than the case of traditional crowd sourcing.
We aim to introduce the potential of the community relationship as better incentive than money,
in terms of working performance and efficiency.
We are now building SOPSOP, a proofreading system for Statement of Purpose (\acr{SOP}), 
to utlize the crowd power of online group for discussing SOP.
The primitive findings of the ongoing project is now submitted to \textit{CSCW '15} as a poster,
and we are still working on it for further exploration. \\




