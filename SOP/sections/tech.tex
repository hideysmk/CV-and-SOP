\noindent 
I worked on \textbf{Social Signal Processing (\acr{SSP})} under the advisement of Professor Jane Yung-Jen Hsu. 
When pursuing master's degree in computer science at National Taiwan University (\acr{NTU}), 
I proposed a framework to improve the process of \textbf{Conversational Engagement Analysis}. 
Since psychological theories are mostly based on observable behaviors, 
a gap exists between the numerical sensor data and the interpretation of social activities.
Instead of direct assessing engagement from raw sensor data,
interpreting behaviors patterns first can place the analysis on a more concrete basis of behavioral science. 
With Coupled Hidden Markov Model (\acr{CHMM}) presenting temporal information, 
the machine-assessed engagement level received an 82\% accuracy on expert-labeled data. 
The early version of the work was accepted by \textit{AAAI-12 Activity Context Representation Workshop}. 
Being accepted by the research community, I first realized that I am qualified to deliver concrete research results. 
The chance of presenting and discussing the research idea to global researchers also inspired me to conduct following research, 
which led to the final version of my master’s thesis.\\

\noindent 
To augment CMC, I explored audio channel under the supervision of Professor Ted Selker. 
In 2012 summer, Prof. Selker recruited five students from NTU for short-term research.
I was the only student selected to join the \textit{CAMEO} project, working on designing \textbf{Intelligent Conference System}. 
We used auditory icons to present three social information: speaker identity, presence, and entry/exit. 
I carefully designed the icons according to syntax (sound and placement) and semantic (relationship to the conversational channel). 
Successfully showing that utilizing audio channel can improve users' sensation of the interaction, the paper was accepted by \textit{INTERACT '13}. 
In addition to the achievement of research, 
the experience also helped me develop the ability of overcoming the challenge of facing new research topic with new colleagues in a new environment. 
After the CMUSV days, I am more confident that pursuing PhD will be not only necessary but also a correct choice to my life.\\

\noindent
Currently, I am working on \textbf{Community Sourcing} with Professor Hao-Chuan Wang.
In community sourcing, 
requesters and workers share common traits and have closer relationship than the case of traditional crowd sourcing.
We aim to introduce the potential of the community relationship as better incentive than money,
in terms of working performance and efficiency.
We are now building SOPSOP, a proofreading system for Statement of Purpose (\acr{SOP}), 
to utlize the crowd power of online group for discussing SOP.
The primitive findings of the ongoing project is now submitted to \textit{CSCW '15} as a poster,
and we are still working on it for further exploration. \\




