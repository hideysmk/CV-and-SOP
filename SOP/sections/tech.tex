\begin{wrapfigure}{R}{0.3\textwidth}
  \begin{center}
  \includegraphics[width=0.2\textwidth]{framework.pdf}
  \caption{}
  \label{fig:framework}
  \end{center}
\end{wrapfigure}

\noindent
Modern computing devices are no more ``just one of communication channels''.
Intelligent and ubiquitous computing make computers possible to process information for communication participants,
and actively participate in the interaction.
I would describe these new machines as ``implicit participants'',
and the interaction is the interplay between mulitple human and multiple comptuers.
However, social mediation that current computing devices can actively provide are still limited.
My prior experiences has provided me several pieces of the puzzle,
but to collect more pieces and aggregate them into a whole picture, 
pursuing a PhD is necessary to me.\\

\noindent
As the first step, I started working on crafting intelligent machines to understand humans' mind.
I started working on social signal processing of \textit{Face-to-Face Conversational Engagement} when pursuing Master's degree,
advised by Prof. Jane Hsu in Intelligent Agents Lab, National Taiwan University (\acr{NTU}).
The traditional approach of social signal analysis is using sophisticated mathematical model to interpret social behaviors from digital signals.
I proposed a framework which are flexible to involve human intelligence for assessing face-to-face conversational engagement.
With Coupled Hidden Markov Model, 
the machine-assessed engagement level received an 82\% accuracy on expert-labeled data. 
The work was accepted by \textit{AAAI-12 Activity Context Representation Workshop}, and developed into my Master's thesis eventually.\\
% Oct. 28: remember to modify this part as to add "crowds' power"

\noindent
Realizing that in addition to high-functional computing machines, 
elegant interaction design is also vital for forming a computer-mediated communication.
I joined the \textit{Reminiscence-Aiding Interface} project, 
which is a cooperative project with Prof. Rung-Huei Liang,
to absorb knowledge of  
I and my colleagues explored the potential effects of soundscape in aiding reminiscence,
i.e. recall and telling the past stories.
The series of the soundscape project were accepted as posters in \textit{DIS '12} and \textit{CHI '13}, 
and full paper in \textit{IASDR '13}, a top conference in design research.\\

\noindent
As the most active lab member,
Prof. Jane Hsu recommended me to visit Carnegie Mellon University Silicon Valley for cooperative research.
I joined a project led by Prof. Ted Selker on \textit{Auditory Presented Social Information}.
We aimed to understand how audio channel help present three pieces of social information: 
speaker identity, presence, and entry/exit, in teleconference system.
I carefully designed the icons according to syntax (sound and placement) and semantic (relationship to the conversational channel).
Successfully showing that information from audio channel can improve users' sensory experience in the interaction, 
the paper was accepted by \textit{INTERACT '13}. 
The knowledge of presenting social information in audio channel thus became the next piece that I have collected. \\


\noindent
\textit{Social-Driven Editing System}:
Currently, I am working on friendsourcing with Prof. Hao-Chuan Wang, who establishes the first social computing lab in Taiwan.
I beleive that social relationship is an excellent incentive to mobilize people for high quality sourcing.
In the pilot study, we found that requesters are willing to pay more than workers expect in friendsourcing,
which is counter-intuitive that people usually ask friends for free solutions.
This primitive finding is now submitted as a poster to \textit{CSCW '15}.
We are now working on designing \textit{Social-Powered SOP} (\acr{SOPSOP}),
a social-driven proofreading and editting system for statement of purpose, aiming to further understand the phonemena. \\

\noindent
After completing one year mandatory military service, I continued my research in Intel-NTU research center. 


















